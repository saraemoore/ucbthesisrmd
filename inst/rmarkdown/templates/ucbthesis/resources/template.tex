\documentclass[$if(degree_short)$$degree_short$,$endif$$for(classoption)$$classoption$$sep$,$endfor$]{ucbthesis}

$if(fontfamily)$
\usepackage[$for(fontfamilyoptions)$$fontfamilyoptions$$sep$,$endfor$]{$fontfamily$}
$else$
\usepackage{lmodern}
$endif$

% Double spacing, if you want it. Do not use for the final copy.
$if(doublespaced)$
\def\dsp{\def\baselinestretch{2.0}\large\normalsize}
\dsp
$endif$

% If the Grad. Division insists that the first paragraph of a section
% be indented (like the others), then include this:
$if(indentfirst)$
\usepackage{indentfirst}
$endif$

%%%%%%%%%%%%%%%%%%
% If you want to use "sections" to partition your thesis:
$if(usesections)$
\counterwithout{section}{chapter}
\setsecnumdepth{subsubsection}
\def\sectionmark#1{\markboth{#1}{#1}}
\def\subsectionmark#1{\markboth{#1}{#1}}
\renewcommand{\thesection}{\arabic{section}}
\renewcommand{\thesubsection}{\thesection.\arabic{subsection}}
\makeatletter
\let\l@subsection\l@section
\let\l@section\l@chapter
\makeatother

\renewcommand{\thetable}{\arabic{table}}
\renewcommand{\thefigure}{\arabic{figure}}
$endif$
%%%%%%%%%%%%%%%%%%

\usepackage{amssymb,amsmath}
\usepackage{fixltx2e} % provides \textsubscript
\usepackage{mathspec}
\defaultfontfeatures{Ligatures=TeX,Scale=MatchLowercase}
$for(fontfamilies)$
  \newfontfamily{$fontfamilies.name$}[$fontfamilies.options$]{$fontfamilies.font$}
$endfor$
$if(mainfont)$
    \setmainfont[$for(mainfontoptions)$$mainfontoptions$$sep$,$endfor$]{$mainfont$}
$endif$
$if(sansfont)$
    \setsansfont[$for(sansfontoptions)$$sansfontoptions$$sep$,$endfor$]{$sansfont$}
$endif$
$if(monofont)$
    \setmonofont[Mapping=tex-ansi$if(monofontoptions)$,$for(monofontoptions)$$monofontoptions$$sep$,$endfor$$endif$]{$monofont$}
$endif$
$if(mathfont)$
    \setmathfont(Digits,Latin,Greek)[$for(mathfontoptions)$$mathfontoptions$$sep$,$endfor$]{$mathfont$}
$endif$
$if(CJKmainfont)$
    \usepackage{xeCJK}
    \setCJKmainfont[$for(CJKoptions)$$CJKoptions$$sep$,$endfor$]{$CJKmainfont$}
$endif$

% use upquote if available, for straight quotes in verbatim environments
\IfFileExists{upquote.sty}{\usepackage{upquote}}{}
% use microtype if available
\IfFileExists{microtype.sty}{%
\usepackage{microtype}
\UseMicrotypeSet[protrusion]{basicmath} % disable protrusion for tt fonts
}{}

\usepackage[unicode=true]{hyperref}
\hypersetup{
$if(title-meta)$
            pdftitle={$title-meta$},
$endif$
$if(author-meta)$
            pdfauthor={$author-meta$},
$endif$
$if(keywords)$
            pdfkeywords={$for(keywords)$$keywords$$sep$; $endfor$},
$endif$
            colorlinks=false,
            pdfborder={0 0 0},
            breaklinks=true}
\urlstyle{same}  % don't use monospace font for urls

$if(natbib)$
\usepackage{natbib}
\bibliographystyle{$if(biblio-style)$$biblio-style$$else$plainnat$endif$}
$endif$

$if(biblatex)$
\usepackage[$if(biblio-style)$style=$biblio-style$,$endif$$for(biblatexoptions)$$biblatexoptions$$sep$,$endfor$]{biblatex}
$for(bibliography)$
\addbibresource{$bibliography$}
$endfor$
$endif$

$if(listings)$
\usepackage{listings}
$endif$

$if(highlighting-macros)$
$highlighting-macros$
$endif$

$if(verbatim-in-note)$
\usepackage{fancyvrb}
\VerbatimFootnotes % allows verbatim text in footnotes
$endif$

$if(tables)$
\usepackage{longtable,booktabs}
$endif$

$if(graphics)$
\usepackage{graphicx,grffile}
\makeatletter
\def\maxwidth{\ifdim\Gin@nat@width>\linewidth\linewidth\else\Gin@nat@width\fi}
\def\maxheight{\ifdim\Gin@nat@height>\textheight\textheight\else\Gin@nat@height\fi}
\makeatother
% Scale images if necessary, so that they will not overflow the page
% margins by default, and it is still possible to overwrite the defaults
% using explicit options in \includegraphics[width, height, ...]{}
\setkeys{Gin}{width=\maxwidth,height=\maxheight,keepaspectratio}
$endif$

$if(links-as-notes)$
% Make links footnotes instead of hotlinks:
\renewcommand{\href}[2]{#2\footnote{\url{#1}}}
$endif$

$if(strikeout)$
\usepackage[normalem]{ulem}
% avoid problems with \sout in headers with hyperref:
\pdfstringdefDisableCommands{\renewcommand{\sout}{}}
$endif$

\setlength{\emergencystretch}{3em}  % prevent overfull lines
\providecommand{\tightlist}{%
  \setlength{\itemsep}{0pt}\setlength{\parskip}{0pt}}

$if(numbersections)$
\setcounter{secnumdepth}{$if(secnumdepth)$$secnumdepth$$else$5$endif$}
$else$
\setcounter{secnumdepth}{0}
$endif$

$if(subparagraph)$
$else$
% Redefines (sub)paragraphs to behave more like sections
\ifx\paragraph\undefined\else
\let\oldparagraph\paragraph
\renewcommand{\paragraph}[1]{\oldparagraph{#1}\mbox{}}
\fi
\ifx\subparagraph\undefined\else
\let\oldsubparagraph\subparagraph
\renewcommand{\subparagraph}[1]{\oldsubparagraph{#1}\mbox{}}
\fi
$endif$

$if(dir)$
  % load bidi as late as possible as it modifies e.g. graphicx
  $if(latex-dir-rtl)$
  \usepackage[RTLdocument]{bidi}
  $else$
  \usepackage{bidi}
  $endif$
$endif$

% additional preamble specifications go here, such as theorems or hyphenation:
$for(header-includes)$
$header-includes$
$endfor$
% (these are examples and can be removed from the yaml header if not used)

\begin{document}

\title{$title$}
\author{$author$}
\degreesemester{$degreesemester$}
\degreeyear{$degreeyear$}
\degree{$degree$}

$if(chair)$
\chair{$chair$}
$endif$

$if(cochair)$
\cochair{$cochair$}
$endif$

$if(cochairs)$
\cochairs{$cochairs$}
$endif$

\othermembers{
$for(othermembers)$
$for(othermembers)$
$othermembers$\\
$endfor$
$endfor$
\vspace{14pt}
}

\numberofmembers{$numberofmembers$}
\field{$field$}

$if(emphasis)$
\emphasis{$emphasis$}
$endif$

$if(jointinstitution)$
\jointinstitution{$jointinstitution$}
$endif$

$if(campus)$
\campus{$campus$}
$endif$

\maketitle

$if(approvalpage)$
\approvalpage
% ^^ DO NOT include this when compiling your final draft
$endif$

\copyrightpage

\begin{abstract}

% If your abstract is complex, place the text here with LaTeX formatting
% and remove the 'abstract' field from the YAML header in the main file.
% If you need to use a \section command in your abstract, you will need
% to use \section*, \subsection*, etc. so that you don't get any numbering.
% Sections are optional but not disallowed in abstracts for a thesis or
% dissertation to be submitted to the Graduate Degrees Office at UC Berkeley.

$if(abstract)$
$abstract$
$endif$

\end{abstract}

$for(include-before)$
$include-before$

$endfor$

\begin{frontmatter}

$if(dedication)$
{
\begin{dedication}
\null\vfil
\begin{center}
$dedication$
\end{center}
\vfil\null
\end{dedication}
}
$endif$

$if(toc)$
	\setcounter{tocdepth}{$toc-depth$}
	\tableofcontents
	$if(lof)$
		\clearpage
		\listoffigures
		$if(lot)$
			\clearpage
			\listoftables
		$endif$
	$else$
		$if(lot)$
			\clearpage
			\listoftables
		$endif$
	$endif$
$else$
	$if(lof)$
		\listoffigures
		$if(lot)$
			\clearpage
			\listoftables
		$endif$
	$else$
		$if(lot)$
			\listoftables
		$endif$
	$endif$
$endif$

$if(acknowledgements)$
{
\begin{acknowledgements}
$acknowledgements$
\end{acknowledgements}
}
$endif$

\end{frontmatter}

\pagestyle{headings}

$body$

$if(natbib)$
$if(bibliography)$
$if(biblio-title)$
\renewcommand\bibname{$biblio-title$}
$endif$
\bibliography{$for(bibliography)$$bibliography$$sep$,$endfor$}

$endif$
$endif$

$if(biblatex)$
\printbibliography$if(biblio-title)$[title=$biblio-title$]$endif$

$endif$

$for(include-after)$
$include-after$

$endfor$

\end{document}
